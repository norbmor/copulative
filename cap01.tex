\chapter{Introducción}
\vspace*{-.25in}
\section{Sobre las estructuras copulativas}
Las estructuras copulativas se definen normalmente por ser construcciones con un sujeto, un predicado no verbal y un verbo copulativo o cópula\index{cópula} ~\autocite{Bentley20179Copular-and-}. Aunque no hay un consenso generalizado sobre los límites de la clase de verbos copulativos, se asume que los verbos copulativos son verbos sin argumentos o sin valencia y actúan como mero soporte de la información gramatical de concordancia, tiempo, aspecto y modo y otras nociones.


Copulativas en criollos relacionados con el español~\autocite{Green1997Romance-creoles}. 

Añadir ejemplos



\ex[glspace=!1em,everygla={},everyglb={},aboveglbskip=-.2ex]
\begingl
\gla Melania es feliz //
\glb Melania ser.{\sc presind}.{\sc 3sg} feliz  //
\endgl
\xe
