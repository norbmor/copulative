\chapter{Introducción}
\vspace*{-.25in}
\section{Sobre las estructuras copulativas}
Las estructuras copulativas se definen normalmente por ser construcciones con un sujeto, un predicado no verbal y un verbo copulativo o cópula\index{cópula} ~\autocite{Bentley20179Copular-and-}. Aunque no hay un consenso generalizado sobre los límites de la clase de verbos copulativos, se asume que los verbos copulativos son verbos sin argumentos o sin valencia y actúan como mero soporte de la información gramatical de concordancia, tiempo, aspecto y modo y otras nociones.


Copulativas en criollos relacionados con el español~\autocite{Green1997Romance-creoles}. 

Añadir ejemplos



\ex[glspace=!1em,everygla={},everyglb={},aboveglbskip=-.2ex]
\begingl
\gla Melania es feliz //
\glb Melania ser.{\sc presind}.{\sc 3sg} feliz  //
\endgl
\xe


Añadir un pequeño epígrafe sobre los reducción de la grupos cuantificativos en las oraciones que aparecen con el verbo \textit{ser}.

\pex[*]
\a	Estoy tan cansado que me caigo de sueño -> Estoy que me caigo de sueño.
\a	Están tan emocionados que no hay quien los eché de casa -> Están que no hay quien los eche de casa.
\xe


único sujeto por atributo. Consecuencia de la relación de mando-c mutuo que exige la relación de predicación entre sujeto y atributo ~\autocite{Williams1980Predication,Bowers1993The-Syntax-of-Predic}. En consecuencia en oraciones como las siguientes hay que asumir la existencia de sujetos tácitos que puedan ser sujetos del segundo atributo.

\pex[*]
\a Juan fue despedido por incauto.
\a El castigo fue injusto por excesivo.
\xe


En cambio, el segundo atributo que puede aparecer con algunos participios no requiere un sujeto propio ya que el carácter verbal del participio permite que este se pueda construir con su propios predicativos si sus requisitos semánticos lo permiten~\autocite{Bosque1999el-sintagma-adjetiva}

\pex[*]
\a	El juez estaba acostado vestido.
\a	El ejecución de la obra estaba considerada ejemplar.
\xe
