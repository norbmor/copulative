\chapter{Introducción}
\vspace*{-.25in}
\section{Sobre las estructuras copulativas}
Las estructuras copulativas se definen normalmente por ser construcciones con un sujeto, un predicado no verbal y un verbo copulativo o cópula\index{cópula} ~\autocite{Bentley20179Copular-and-}. Aunque no hay un consenso generalizado sobre los límites de la clase de verbos copulativos, se asume que los verbos copulativos son verbos sin argumentos o sin valencia y actúan como mero soporte de la información gramatical de concordancia, tiempo, aspecto y modo y otras nociones. El verbo copulativo actúa como nexo de unión entre los dos elementos de la relación predicativa, el sujeto y el predicado.

La existencial de la cópula en la mayor parte de las lenguas románicas es incontrovertida así como en aquellas que derivan del indoeuropeo. Las existencia de verbos copulativos tampoco está discutida en las lenguas finoúgrias como el húngaro o el finés ni en las lenguas de ascendencia altaica como el túrco. 

Un gran número de lenguas carecen por completo de elementos copulativos que se utilizen para establacer un enlace entre el sujeto y el atributo. Las lenguas austronesias, las lenguas aborígenes australianas, algunas lenguas dravídicas, y algunas lenguas nigerocongolesas son también lenguas que carecen de cópulas.

Más polémico resulta el estatus de la cópula en lenguas como el hebreo, el árabe, el birmano, el ruso y otras. En esta lenguas la cópula no aparace en algunos tiempos verbales, cuando lo hace, su naturaleza está más cerca de un elemento demostrativo que de un verbo. Este el caso del chino mandarín moderno cuya cópula tiene un origen como pronombre demostrativo bien documentado en chino arcaico~\parentcites{Li1977Mechanismdevelopmentcopula}{Lohndal2009Copula-cycle}{Gelderen2011The-Pronominal}. 


%Copulativas en criollos relacionados con el español~\autocite{Green1997Romance-creoles}. Añadir ejemplos



\ex[glspace=!1em,everygla={},everyglb={},aboveglbskip=-.2ex]
\begingl
\gla Melania es feliz //
\glb Melania ser.{\sc presind}.{\sc 3sg} feliz  //
\endgl
\xe


Añadir un pequeño epígrafe sobre los reducción de la grupos cuantificativos en las oraciones que aparecen con el verbo \textit{ser}.

\pex[*]
\a	Estoy tan cansado que me caigo de sueño $\longrightarrow$  Estoy que me caigo de sueño.
\a	Están tan emocionados que no hay quien los eché de casa $\longrightarrow$  Están que no hay quien los eche de casa.
\xe


El hecho de que haya un único sujeto por atributo. es consecuencia de la relación de mando-c mutuo que exige la relación de predicación entre sujeto y atributo ~\autocite{Williams1980Predication,Bowers1993The-Syntax-of-Predic}. En oraciones como las siguientes hay que asumir la existencia de sujetos tácitos que puedan ser sujetos del segundo atributo.

\pex[*]
\a Melania fue ascendida por competente.
\a El premio fue sorprendente por inesperado.
\xe


En cambio, el segundo atributo que puede aparecer con algunos participios no requiere de un sujeto propio ya que el carácter verbal del participio hace que este se pueda construir con su propios predicativos si sus requisitos semánticos lo permiten~\autocite{Bosque1999el-sintagma-adjetiva}

\pex[*]
\a	El juez estaba acostado vestido.
\a	La ejecución de la obra estaba considerada ejemplar.
\xe

\section{Tipos de oraciones copulativas}


\section{Las oraciones copulativas}

Las oraciones copulativas caracterizadoras pueden aparecer tanto con ser como con estar. Se trata de aquellas oraciones en la que el atributo predica del sujeto cualquier tipo de característica. Así, el atributo puede hacer referencia a cualidades físicas, psíquicas o morales, material, origen o procedencia, posesión o pertenencia o adscripción a una clase~\parencites[][]{Leborans1999Predicacion}[][]{Bentley20179Copular-and-}:

\pex
\a Melania es inteligente.
\a Alfonso está cansado.
\a Elisa es de carácter fuerte.
\a El banco es de madera.
\a Mis tíos son de Madrid.
\a El libro es de Laura.
\a Pedro es camarero
\xe



{}\textcite{Leborans1999Predicacion} señala que estas oraciones comparten varias características de orden sintáctico, semántico y pragmático, como son las siguientes:


\textbf{I}. Generalmente su sujeto es referencial, aunque puede no serlo. Lo relevante, en cualquier caso, es que el atributo no puede nunca ser referencial.

\textbf{II}. El atributo de las copulativas caracterizadoras es conmutable por las proformas \textit{lo} o \textit{eso}:

\pex
\a Alicia es inteligente $\longrightarrow$ Alicia lo es
\a Diego está cansado $\longrightarrow$ Diego lo está
\a Marta es profesora $\longrightarrow$ Marta lo es
\a El cine es arte puro $\longrightarrow$ El cine lo es
\xe




\section{Las copulativas inversas}

Las oraciones copulativas inversas, también llamadas especificativas 
