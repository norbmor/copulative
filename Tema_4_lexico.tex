\documentclass[12pt]{article}
\usepackage[colorlinks,linkcolor=blue,citecolor=blue]{hyperref}
\usepackage[spanish,es-noquoting]{babel}
\usepackage[anythingbreaks]{breakurl}
%\usepackage{pstricks}
%\usepackage{pst-jtree}
%\usepackage{pst-node}
%\usepackage[utf8]{inputenc}
%\usepackage[expert,uprightgreek,altbullet]{lucidabr}
\usepackage[T1]{fontenc}
\usepackage{textcomp}
\usepackage{graphicx}
\usepackage{fancyhdr}
\usepackage{tipa} 
\usepackage[headheight=20pt,tmargin=2cm,bmargin=2cm,lmargin=2.5cm,rmargin=2.5cm]{geometry}
\usepackage{expex}
\usepackage{parskip}
\usepackage[autostyle]{csquotes}
\usepackage{ctable,booktabs}
\usepackage{multirow}
\usepackage{makecell}
\usepackage[sort]{natbib}
\usepackage{enumitem}
\everymath={\rm}
\bibpunct{[}{]}{;}{a}{}{,}

 \DeclareOldFontCommand{\rm}{\normalfont\rmfamily}{\mathrm}

\definecolor{fond}{RGB}{240,240,240}
\newcommand{\otoprule}{\midrule[\heavyrulewidth]}

\def\TO{\quad$\rightarrow$\quad}

\pagestyle{fancy}

 \lhead{\footnotesize Lengua española I\\
Grado en Humanidades\\
Doble Grado en Humanidades y Magisterio de Educación Primaria
} \chead{} \rhead{}

\lfoot{\texttt{norberto.morenoquibe@uah.es}} \cfoot{} \rfoot{\thepage}

\renewcommand{\headrulewidth}{0.4pt}
\renewcommand{\footrulewidth}{0.4pt}

\def\TO{\quad$\rightarrow$\quad}

\title{Tema 4. El léxico}


\author{Norberto Moreno Quibén\\
        \small $<$norberto.morenoquibe@uah.es$>$}

\date{}

\begin{document}
%\maketitle
\textbf{\Large Tema 4. El léxico}

\tableofcontents

Material elaborado a partir de~\cite{bosque1982sobre-la-teoria,bosque2004combinatoria-y-signi,murphy2010lexicalmeaning,jezek2015the-lexicon-an,pustejovsky2019the-lexicon} y ~\cite[caps. «Lexicología» y «Diccionarios»][]{2016enciclopedia-de-linguistica}.
%\cite{Jezek2015The-lexicon-an}
%\cite{Bosque1982Sobre-la-teoria}
%\cite[cap. «Lexicología»]{2016Enciclopedia-de-Linguistica}
%\cite{Bosque2004Combinatoria-y-signi}

%\bibliographystyle{linquiry2esp}
%\bibliography{BibFileTGR}
\printbibliography

\section{El lexicon}
\begin{itemize}
  \item El \href{https://dle.rae.es/?w=diccionario}{\psframebox{DLE}} de la RAE contiene en su última edición 93 111 entradas.
\end{itemize}


El \textbf{lexicón} como diccionario interno o mental, el \textbf{lexicón mental}.
\begin{itemize}
  \item El lexicón mental de un hablante medio contiene 250 000 entradas.
\end{itemize}



\section{¿Qué es una palabra?}
\section{El significado de las palabras}

El \textbf{lexicón} como diccionario externo, el \textbf{diccionario} o \textbf{lexis}.

\subsection{Intensión y extensión}

La \textbf{intensión} de un concepto proporciona su significado. El conjunto de propiedades, atributos o rasgos que lo distinguen de los demás conceptos.

La \textbf{extensión} de un concepto es en cambio el conjunto de seres a los que se aplica.

En el caso de la palabra \textit{casa}, la intensión del concepto expresado por ella consiste en el conjunto de propiedades que describen lo que es una casa. Coincide más o menos con la definición que aparece en un diccionario.

La extensión del concepto \textit{casa} es, por el contrario, el conjunto de casas pasadas, presentes y futuras.

\subsection{Significado literal y significado figurado}
\section{Relaciones paradigmáticas}
\subsection{Relaciones de equivalencia}
\subsection{Relaciones de inclusión}
\subsection{Relaciones de oposición}
\section{Relaciones sintagmáticas}
\section{El lexicón y los diccionarios}
\subsection{El diccionario y la enciclopedia}
\subsection{Tipos de diccionarios}
\subsection{Estructura de los diccionarios}
\subsection{La definición en los diccionarios}

\bibliographystyle{linquiry2esp}
\bibliography{result-reduced.bib}
\end{document}
